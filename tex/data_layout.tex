\documentclass[a4paper,10pt]{article}

\usepackage[english]{babel}
\usepackage[T1]{fontenc}
\usepackage[ansinew]{inputenc}

\usepackage{lmodern}	% font definition
\usepackage{amsmath}	% math fonts
\usepackage{amsthm}
\usepackage{amsfonts}

\usepackage{tikz}

%%%<
\usepackage{verbatim}
\usepackage[active,tightpage]{preview}
\PreviewEnvironment{tikzpicture}
\setlength\PreviewBorder{5pt}%
%%%>

\usetikzlibrary{decorations.pathmorphing} % noisy shapes
\usetikzlibrary{fit}					% fitting shapes to coordinates
\usetikzlibrary{backgrounds}	% drawing the background after the foreground

\begin{document}

\begin{figure}[htbp]
\centering
% The state vector is represented by a blue circle.
% "minimum size" makes sure all circles have the same size
% independently of their contents.
\tikzstyle{ann} = [fill=white,inner sep=1pt]
\tikzstyle{state}=[circle,
                                    thick,
                                    minimum size=1.2cm,
                                    draw=blue!80,
                                    fill=blue!20]

% The measurement vector is represented by an orange circle.
\tikzstyle{measurement}=[circle,
                                                thick,
                                                minimum size=1.2cm,
                                                draw=green!80,
                                                fill=green!25]

% The control input vector is represented by a purple circle.
\tikzstyle{input}=[circle,
                                    thick,
                                    minimum size=1.2cm,
                                    draw=red!80,
                                    fill=red!20]

% The input, state transition, and measurement matrices
% are represented by gray squares.
% They have a smaller minimal size for aesthetic reasons.
\tikzstyle{matrx}=[rectangle,
                                    thick,
                                    minimum size=1cm,
                                    draw=gray!80,
                                    fill=gray!20]

% The system and measurement noise are represented by yellow
% circles with a "noisy" uneven circumference.
% This requires the TikZ library "decorations.pathmorphing".
\tikzstyle{noise}=[circle,
                                    thick,
                                    minimum size=1.2cm,
                                    draw=yellow!85!black,
                                    fill=yellow!40,
                                    decorate,
                                    decoration={random steps,
                                                            segment length=2pt,
                                                            amplitude=2pt}]

% Everything is drawn on underlying gray rectangles with
% rounded corners.
\tikzstyle{background}=[rectangle,
                                     fill=gray!0,
                                     inner sep=0.2cm,
                                     rounded corners=5mm]

\begin{tikzpicture}[>=latex,text height=1.5ex,text depth=0.25ex]
    % "text height" and "text depth" are required to vertically
    % align the labels with and without indices.
  
  % The various elements are conveniently placed using a matrix:
  \matrix[row sep=0.5cm,column sep=0.5cm] {
    % First line: Control input
    		\node (0_0) [input]{}; &
        \node (0_1) [measurement]{1};     &
        \node (0_2) [state]{}; &
        \node (0_3) [measurement]{2}; &
        \node (0_4) [input]{};     &
        \node (0_5) [measurement]{3}; &
        \node (0_6) [state]{}; &
        \node (0_7) [measurement]{4};     &
        \node (0_8) [input]{}; &
        \node (0_9) [measurement]{5}; &
        \node (0_10) [state]{};     &
        \node (0_11) [measurement]{6}; &
        \node (0_12) [input]{}; &
        \node (0_13) [measurement]{7}; &
        \node (0_14) [state]{};     &
        \node (0_15) [measurement]{8}; &
        \node (0_16) [input]{}; &
        \node (0_17) [measurement]{9};     &
        \node (0_18) [state]{}; &
        \node (0_19) [measurement]{10}; &
        \node (0_20) [input]{};     &
        \\
        \node (1_0) [measurement]{11};     &
        &
        \node (1_2) [measurement]{12};     &
        &
        \node (1_4) [measurement]{13};     &
        &
        \node (1_6) [measurement]{141};     &
        &
        \node (1_8) [measurement]{15};     &
        &
        \node (1_10) [measurement]{16};     &
        &
        \node (1_11) [measurement]{17};     &
        &
        \node (1_13) [measurement]{18};     &
        &
        \node (1_15) [measurement]{19};     &
        &
        \node (1_17) [measurement]{20};     &
        &
        \node (1_19) [measurement]{21};     &
        \\
        \node (2_0) [input]{}; &
        \node (2_1) [measurement]{22};     &
        \node (2_2) [state]{}; &
        \node (2_3) [measurement]{23}; &
        \node (2_4) [input]{};     &
        \node (2_5) [measurement]{24}; &
        \node (2_6) [state]{}; &
        \node (2_7) [measurement]{25};     &
        \node (2_8) [input]{}; &
        \node (2_9) [measurement]{26}; &
        \node (2_10) [state]{};     &
        \node (2_11) [measurement]{27}; &
        \node (2_12) [input]{}; &
        \node (2_13) [measurement]{28}; &
        \node (2_14) [state]{};     &
        \node (2_15) [measurement]{29}; &
        \node (2_16) [input]{}; &
        \node (2_17) [measurement]{30};     &
        \node (2_18) [state]{}; &
        \node (2_19) [measurement]{31}; &
        \node (2_20) [input]{};     &
        \\
        \node (3_0) [measurement]{32};     &
        &
        \node (3_2) [measurement]{33};     &
        &
        \node (3_4) [measurement]{34};     &
        &
        \node (3_6) [measurement]{35};     &
        &
        \node (3_8) [measurement]{36};     &
        &
        \node (3_10) [measurement]{37};     &
        &
        \node (3_11) [measurement]{38};     &
        &
        \node (3_13) [measurement]{39};     &
        &
        \node (3_15) [measurement]{40};     &
        &
        \node (3_17) [measurement]{41};     &
        &
        \node (3_19) [measurement]{42};     &
        \\
        \node (4_0) [input]{}; &
        \node (4_1) [measurement]{43};     &
        \node (4_2) [state]{}; &
        \node (4_3) [measurement]{44}; &
        \node (4_4) [input]{};     &
        \node (4_5) [measurement]{45}; &
        \node (4_6) [state]{}; &
        \node (4_7) [measurement]{46};     &
        \node (4_8) [input]{}; &
        \node (4_9) [measurement]{47}; &
        \node (4_10) [state]{};     &
        \node (4_11) [measurement]{48}; &
        \node (4_12) [input]{}; &
        \node (4_13) [measurement]{49}; &
        \node (4_14) [state]{};     &
        \node (4_15) [measurement]{50}; &
        \node (4_16) [input]{}; &
        \node (4_17) [measurement]{51};     &
        \node (4_18) [state]{}; &
        \node (4_19) [measurement]{52}; &
        \node (4_20) [input]{};     &
        \\
    };
    
    
    % Now that the diagram has been drawn, background rectangles
    % can be fitted to its elements. This requires the TikZ
    % libraries "fit" and "background".
    % Control input and measurement are labeled. These labels have
    % not been translated to English as "Measurement" instead of
    % "Messung" would not look good due to it being too long a word.
    \begin{pgfonlayer}{background}
        \node [background,
                    fit=(0_0) (4_20)] {};
    \end{pgfonlayer}
    \draw[arrows=<->,line width=.4pt](-18,4.6)--(18,4.6);
    \draw[arrows=<->,line width=.4pt](-18.5,4.8)--(-18.5,-4.8);
    \node[ann] at (0,4.5) {\huge{22}};
    \node[ann] at (-18.5,0) {\huge{5}};
\end{tikzpicture}

\caption{Kalman filter system model}
\end{figure}

\end{document}